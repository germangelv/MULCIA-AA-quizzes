\documentclass{article}
\usepackage[spanish]{babel}

% Set page size and margins
% Replace `letterpaper' with`a4paper' for UK/EU standard size
\usepackage[a4paper,top=2cm,bottom=2cm,left=2cm,right=2cm,marginparwidth=2cm]{geometry}

\usepackage{amsmath}
\usepackage{graphicx}
\usepackage{amsfonts}
\usepackage{amssymb}
\usepackage[colorlinks=true, allcolors=blue]{hyperref}
\title{\textbf{MÁSTER UNIVERSITARIO EN
LÓGICA, COMPUTACIÓN E INTELIGENCIA ARTIFICIAL}}
\date{}
\begin{document}

\maketitle
\begin{flushleft}
\textbf{Aprendizaje Automático}
\\\textbf{Apellidos:} Lorenz Vieta
\\\textbf{Nombre:} Germán
\end{flushleft}


\section{Ejercicio}
Sea X un universo y D un conjunto de entrenamiento sobre X. Sean \(E^+ = \{ x \in X \mid(x,1) \in D\}\)  y \(E^- = \{ x \in X \mid(x,0)\in D \}\) . Sea H el conjunto de hipótesis que contiene a todas las hipótesis sobre X. Sea VS el espacio de versiones para (D,H). Sea \( x_0 \in X / x_0 \notin E^+ \cup E^- \) y sea \(h \in VS/h(x_0) = 0\). Demostrar \(h^` \in VS/h^`(x_0) = 1\).\\\\
Considerando la hipótesis \(h^`\in VS\) definida como:
\(h^`(x) =
\left\{
	\begin{array}{ll}
		1  & \mbox{si } x = 1 \\
		h(x) & \mbox{si } x \in X - \{x_0\}
	\end{array}
\right.
\)\\\\
 Por definición \(h^`(x) = 1\).Veamos que \(h^` \in VS\). 
 \begin{flushleft}Como \(H\) es el conjunto de todas las hipótesis, \(h^` \in H\). Además \(h^`(x) = h(x)\) \(\forall x \neq x_0\). En particular, \(h^`(x) = h(x)\) \(\forall x \in E^+ \cup E^-\). Luego, si \(h \in VS \Rightarrow h^`\in VS\)
 \end{flushleft}
 \begin{center}
 Por lo tanto, \(\exists h^` \in VS / h^`(x_0) = 1\)
 \end{center}
\section{Ejercicio}
Sea \(U\) un universo finito y \(C = 2^U\) el conjunto de los objetivos. Sea \(H\) un conjunto de hipótesis sobre \(U\) y \(L\) un algoritmo de aprendizaje tal que su dominio es \( \cup_{c\in C}\cup_{m\geq1}S(m,c)\). Demostrar que si \(H \neq 2^U\), entonces \(L\) no es consistente.\\\\
Sea el algoritmo de aprendizaje \(L=\cup_{c\in C}\cup_{m\geq1}S(m,c)\Rightarrow H \neq 2^U\). Consideremos el vector \(s\in Dom(L)\) con \(L(s)=h \notin 2^U\) con \(s \equiv ((u_1,b_1), \ldots, (u_m,b_m)), u_i \in U, b_i \equiv \{0,1\}\).\\\\
\(L\) es consistente si \(\forall s\in Dom(L), L(s)=h\) es consistente con \(s\). Ademas, \(L(s)=h\) es consistente con \(s\) si \(\forall i \in \{1,\ldots,m\}\) se tiene que \(h(u_i)=b_i\). Pero, por definicion, \(H \neq 2^U\), i.e., \(\exists h \in H\neq 2^U, \exists i \in \{1,\ldots,m\}/h(u_i)\neq b_i\).
 \begin{center}
 Por lo tanto, \(L\) no es consistente.  
 \end{center}
\section{Ejercicio}
Sea \(D =\{ \langle x_1,c(x_1)\rangle,\ldots, \langle x_n,c(x_n)\rangle\}\) un conjunto de entrenamiento para un concepto \(C\) y sea \(H\) un conjunto de hipótesis. Demostrar que el resultado de aplicar el algoritmo de \textbf{Eliminación
de Candidatos} es el mismo para cualquier ordenación de los elementos de \(D\)\\\\
Sean dos ordenadores del conjunto \(D\) y sean \(G^1, S^1\) y \(G^2, S^2\) las cotas generales obtenidas por eliminación de candidatos para cada una de las ordenaciones.\\\\
Supongamos, por reducción al absurdo, que \(G^1 \neq G^2\) y \(S^1 \neq S^2\).
Si \(G^1 \neq G^2\), entonces \(\exists\) hipótesis \(h  \in H / h \in G^1\) y \(h \notin G^2\)\\\\
Si \(h \in G^1\), h es consistente con \(\langle x_1,c(x_1)\rangle\) \(i = 1, \ldots, n\) cuando \(c(x_i)=1\)\\
Si \(h \notin G^1\), h no es consistente con \(\langle x_1,c(x_1)\rangle\) para algún \(i = 1\ldots, n\) cuando \(c(x_i)=1\)\\\\
Cuando \(c(x_i)=0\), se eliminaran de las cotas generales \(G^1, G^2\) las hipótesis "menos" generales, y se incluirán  las especializaciones minimales que sean consistentes con \(x_i\), siguiendo el razonamiento anterior se tiene:\\\\
Si \(G^1 \neq G^2 \Rightarrow h\) consistente con \(\langle x_1,c(x_1)\rangle  \forall i = 1, \ldots,n\) y \(h\) inconsistente con \(\langle x_1,c(x_1)\rangle\) para algún \(i \in \{1, \ldots,n\}\).\\\\
Esto es una contradicción, luego \(G^1 = G^2\). De la misma forma se puede probar para la cota especifica \(S\).

Si \(c(x_i)=1, S_1\) y \(S_2\) estarán formadas por generalizaciones minimales que sean consistentes con \(x_i\) y se eliminaran las menos especificas.\\\\
Si \(S_1 \neq S_2 \Rightarrow \exists h \in S^1\) consistente con \(\langle x_1,c(x_1)\rangle \forall i = 1, \ldots,n\) y \(h \notin S^2\) inconsistente con \(\langle x_1,c(x_1)\rangle\) para algún \(i\in \{1, \ldots,n\}\).\\\\
Así, \(S^1 =  S^2\). Luego las cotas generales y especificas son las mismas para ambas ordenaciones.
 \begin{center}
Por lo tanto, el resultado obtenido por eliminación de candidatos no depende de la ordenación.   
 \end{center}

\section{Ejercicio}
Aplica los algoritmos de \textit{aprendizaje por enumeración} y \textit{Find-S} para los siguientes problemas de aprendizaje:
\subsection{Problema}
\begin{flushleft}
\(X = \mathbb R^2\)\\
\(H = \{h_n : X \Rightarrow \{0, 1\} \mid n \in \mathbb N \land h_n((x, y)) = 1 \Leftrightarrow x^2 + y^2 \leq n^2\}\)\\
\(s = \{ \langle (1, 1), 1\rangle , \langle (3, 4), 1\rangle , \langle (2, 2), 1\rangle , \langle(4, 7), 0\rangle ,\}\)
\end{flushleft}

\textbf{Aprendizaje por enumeración}: Recorrer el conjunto de hipótesis buscando \(n / h_n (x_i) = c(x_i) \forall i = 1, 2, 3, 4\).

\begin{itemize}
    \item \( h_1 : h_1 ((x, y)) = 1 \Leftrightarrow x^2 + y^2 \leq 1
    \\ \langle(1, 1), 1\rangle : h_1 ((1, 1)) \neq 1 = c((1, 1))\)
    \item \( h_2 : h_2 ((x, y)) = 1 \Leftrightarrow x^2 + y^2 \leq 4
    \\ \langle(1, 1), 1\rangle : h_2 ((1, 1)) = 1 = c((1, 1))
    \\ \langle(3, 4), 1\rangle : h_2 ((3, 4)) \neq 1 = c((3, 4))\)
    \item \( h_3 : h_3 ((x, y)) = 1 \Leftrightarrow x^2 + y^2 \leq 9
    \\ \langle(1, 1), 1\rangle : h_3 ((1, 1)) = 1 = c((1, 1))
    \\ \langle(3, 4), 1\rangle : h_3 ((3, 4)) \neq 1 = c((3, 4))\)
    \item \( h_4 : h_4 ((x, y)) = 1 \Leftrightarrow x^2 + y^2 \leq 16
    \\ \langle(1, 1), 1\rangle : h_4 ((1, 1)) = 1 = c((1, 1))
    \\ \langle(3, 4), 1\rangle : h_4 ((3, 4)) \neq 1 = c((3, 4))\)
    \item \( h_5 : h_5 ((x, y)) = 1 \Leftrightarrow x^2 + y^2 \leq 25
    \\ \langle(1, 1), 1\rangle : h_5 ((1, 1)) = 1 = c((1, 1))
    \\ \langle(3, 4), 1\rangle : h_5 ((3, 4)) = 1 = c((3, 4))
    \\ \langle(3, 4), 1\rangle : h_5 ((2, 2)) = 1 = c((2, 2))
    \\ \langle(4, 7), 1\rangle : h_5 ((4, 7)) = 0 = c((4, 7))\)
    \\ idem para \(h_6, h_7, h_8\)
    \item \( h_9 : h_9 ((x, y)) = 1 \Leftrightarrow x^2 + y^2 \leq 89
    \\ \langle(1, 1), 1\rangle : h_9 ((1, 1)) = 1 = c((1, 1))
    \\ \langle(3, 4), 1\rangle : h_9 ((3, 4)) = 1 = c((1, 1))
    \\ \langle(3, 4), 1\rangle : h_9 ((2, 2)) = 1 = c((2, 2))
    \\ \langle(4, 7), 1\rangle : h_9 ((4, 7)) = 1 \neq c((4, 7))\)
\end{itemize}
\begin{center}
Por lo tanto, \(H = \{h_5,h_6,h_7,h_8\}\)
\end{center}

\textbf{Find-S}: Partiendo de la hipótesis \(h_n\) más específica de \(H\), para cada ejemplo positivo si \(h_n (x_i) = 1\), no hacer nada; en caso contrario reemplazar \(h_n\) por una menor generalización \(h_n\) con \(h_n(x_i) = 1\). Los ejemplos negativos se ignoran.

\begin{itemize}
    \item Paso 0: \(h_0\) es la hipótesis más específica de \(H\)
    \\ \(H = \{h_0 : X \Rightarrow \{0, 1\} \mid h_0 ((x, y)) = 1 \Leftrightarrow  x^2 + y^2 \leq 0 \}\)
    \item Paso 1: \(\langle(1, 1), 1\rangle\)
    \\ \(h_0 ((1, 1)) = 0 \Rightarrow\) se reemplaza \(h_0\) por \(h_2\) ya que \(h_1\) tampoco lo cumple 
    \\ \(h_2 ((1, 1)) = 1\)
    \item Paso 2: \(\langle(3, 4), 1\rangle\)
    \\ \(h_2((3, 4)) = 0 \Rightarrow\) se reemplaza \(h_2\) por \(h_5\) ya que \(h_3\) y \(h_4\) tampoco lo cumplen
    \item Paso 3: \(\langle(2, 2), 1\rangle\)
    \\ \(h_5((2, 2)) = 1 \Rightarrow\) se mantiene \(h_5\)
    \item Paso 4: \(\langle(4, 7), 0\rangle\)
    \\ Al ser un ejemplo negativo se ignora.
\end{itemize}
En este caso la hipótesis de máxima especificidad consistente con todos los ejemplos positivos es \(h_5\):
\begin{center}
Por lo tanto, \(H = \{h_5 : X \Rightarrow \{0, 1\} \mid h_5 ((x, y)) = 1 \Leftrightarrow  x^2 + y^2 \leq 25 \}\)
\end{center}


\subsection{Problema}
\begin{flushleft}
\(X = \mathbb R^2\)\\
\(H = \{h_n \mid n \in \mathbb N\}\) con \(h_0 = \emptyset \) y si \(n \geq 1\), entonces \(h_n = \{(x, y) \in X \mid a, b \in \mathbb N, a \leq x < b, n = \frac{b(b-1)}{2}+ a + 1\}\)\\
\(s = \{\langle(0, 0), 0\rangle, \langle(3, 4), 1\rangle, \langle(2, 2), 1\rangle \}\)
\end{flushleft}

\textbf{Aprendizaje por enumeración}: 

\begin{itemize}
    \item \(h_0 : h_0 = \emptyset\)
    \item \(h_1 : h_1 = \{(x, y) \in X \mid a, b \in \mathbb N, a \leq x < b, 1 = \frac{b(b-1)}{2}+ a + 1\}
    \\ \langle(0, 0), 0\rangle : a \leq 0 < b\). Como \(a \in \mathbb N, h_1 ((0, 0)) = 0 = c(0, 0)
    \\ \langle(3, 4), 1\rangle : a \leq 3 < b, 1 = \frac{b(b-1)}{2}+ 1 +1 \Rightarrow \frac{b(b-1)}{2} + a = 0\)
\begin{itemize}
    \item Para \(a = 1: \frac{b(b-1)}{2} +1 = 0 \Rightarrow b^2 - b + 2 = 0 \Rightarrow \nexists b \in \mathbb N / b^2 - b + 2 = 0\)
    \item Idem para \(a = 2\) y \(a = 3\)
    \item \(h_1 ((3, 4)) = 0 \neq c(3, 4)\)
\end{itemize}
    \item \(h_2 : h_2 = \{(x, y) \in X \mid a, b \in \mathbb N, a \leq x < b, 2 = \frac{b(b-1)}{2}+ a + 1\}
    \\ \langle(0, 0), 0\rangle : a \leq 0 < b\). Como \(a \in \mathbb N, h_1 ((0, 0)) = 0 = c(0, 0)
    \\ \langle(3, 4), 1\rangle : a \leq 3 < b, 2 = \frac{b(b-1)}{2}+ 1 +1 \Rightarrow \frac{b(b-1)}{2} + a = 0\)
\begin{itemize}
    \item Para \(a = 1: \frac{b(b-1)}{2} = 0 \Rightarrow b = 0 \notin \mathbb N, b = 1 \ngtr 3\)
    \item Para \(a =2\) y \(a=3, \nexists b \in \mathbb N / \frac{b(b-1)}{2}+a-1 = 0\)
    \item \(h_2((3, 4)) = 0 \neq c(3, 4)\)
\end{itemize}
    \item Buscamos una generalización: a \(\leq x < b\) con x \(\in \{2, 3\}\), ya que siempre tendremos que se cumple para \(\langle(0, 0), 0\rangle\) porque \(0 \notin \mathbb N\).
    Tenemos: \(a \leq 3 < b \Rightarrow a \in \{1, 2, 3\}\)
    \\\(n = \frac{b(b-1)}{2}+a+1 \Rightarrow b^2 -b+2a+2-2n = 0 \Rightarrow b = \frac{1\pm\sqrt{1-4(2a+2-2n)}}{2}\)
    \\Para que \(b\) tenga solución, \(1 -4(2a + 2 - 2n) \geq 0 \Leftrightarrow n \geq \frac{7+8a}{8}\).
    \begin{itemize}
        \item Para a = 1: \(b = \frac{1\pm\sqrt{1-4(2+2-2n)}}{2} = \frac{1\pm\sqrt{-15+8n)}}{2}\). Como \(b \in \mathbb N\) y \(b > 3:\frac{1\pm\sqrt{-15+8n)}}{2} > n > 5\). Por lo tanto, para existir solución \(n > 5\) y \(\frac{1\pm\sqrt{-15+8n)}}{2} \in \mathbb N\).
        \item Para a = 2: \(b = \frac{1\pm\sqrt{1-4(4+2-2n)}}{2} = \frac{1\pm\sqrt{-23+8n)}}{2}\). Como \(b \in \mathbb N\) y \(b > 3:\frac{1\pm\sqrt{-23+8n)}}{2} > n > 6\). Por lo tanto, para existir solución \(n > 6\) y \(\frac{1\pm\sqrt{-23+8n)}}{2} \in \mathbb N\).
        \item Para a = 3: \(b = \frac{1\pm\sqrt{1-4(6+2-2n)}}{2} = \frac{1\pm\sqrt{-31+8n)}}{2}\). Como \(b \in \mathbb N\) y \(b > 3:\frac{1\pm\sqrt{-31+8n)}}{2} > n > 7\). Por lo tanto, para existir solución \(n > 7\) y \(\frac{1\pm\sqrt{-31+8n)}}{2} \in \mathbb N\).
    \end{itemize}
    \item \(h_n ((x, y)) = 1\) si \(\exists(a, b) \in \mathbb N\) con \(a \leq x < b, n = \frac{b(b-1)}{2}+ a + 1\), luego:
\begin{center}
Por lo tanto, \(H = \{h_n \mid n \in \mathbb N, n > 5, \frac{1+\sqrt{-15+8n}}{2} \in \mathbb N\} \cup \{h_n \mid n \in \mathbb N, n > 6, \frac{1+\sqrt{-23+8n}}{2} \in \mathbb N\} \cup \{h_n \mid n \in \mathbb N, n > 7, \frac{1+\sqrt{-31+8n}}{2} \in \mathbb N\}
\)
\end{center}
\end{itemize}

\textbf{Find-S}: 
\begin{itemize}
    \item Paso 0: \(h_0\) es la hipótesis más especifica de H.
    \item Paso 1: \(\langle(0, 0), 0\rangle\). Al ser un ejemplo negativo se ignora.
    \item Paso 2: \(\langle(3, 4), 1\rangle\).Usare los resultados obtenidos en el algoritmo de aprendizaje por enumeración para facilitar el cálculo. Para que haya solución, \(n\) debe ser, al menos, mayor que 5.
    \begin{itemize}
        \item \(h_6 = \{(x, y) \in X \mid a, b \in \mathbb N, a \leq x < b, 6 = \frac{b(b-1)}{2} + a + 1\}\)
        \begin{itemize}
            \item Para \(a = 1: a \leq 3 < b, 6 = \frac{b(b-1)}{2} + 1 + 1 \Rightarrow b^2 + b - 4 = 0 \Rightarrow b = \frac{1\pm\sqrt{33}}{2} \notin \mathbb N\).
            \item Igual para \(a = 2\) y \(a = 3\).
        \end{itemize}
        \item \(h_7 = \{(x, y) \in X \mid a, b \in \mathbb N, a \leq x < b, 7 = \frac{b(b-1)}{2} + a + 1\}\)
        \begin{itemize}
            \item Para \(a = 1: a \leq 3 < b, 7 = \frac{b(b-1)}{2} + 1 + 1 \Rightarrow b^2 + b - 10 = 0 \Rightarrow b = \frac{1\pm\sqrt{41}}{2} \notin \mathbb N\).
            \item Igual para \(a = 2\) y \(a = 3\).
        \end{itemize}
        \item \(h_8 = \{(x, y) \in X \mid a, b \in \mathbb N, a \leq x < b, 8 = \frac{b(b-1)}{2} + a + 1\}\)
        \begin{itemize}
            \item Para \(a = 1: a \leq 3 < b, 8 = \frac{b(b-1)}{2} + 1 + 1 \Rightarrow b^2 + b - 12 = 0 \Rightarrow b = \frac{1\pm\sqrt{49}}{2} \Rightarrow = 4 \in \mathbb N\) y \(4 > 3\).
            Luego, \(\exists(a,b) = (1,4)\in \mathbb N / a \leq 3 < b, n= \frac{b(b-1)}{2}+a+1\).
        \end{itemize}
    \item \(h_n ((3, 4)) = 0, n = 1, \ldots, 7 \Rightarrow\) se reemplaza la hipótesis por \(h_8\) cumpliendo que \(h_8((3, 4)) = 1 = c(3, 4)\).
    \end{itemize}
\item Paso 3: \(\langle(2, 2), 1\rangle\)
\\\(h_8 ((2, 2)) = 1\) para \(a = 1, b = 4 \Rightarrow\) se mantiene \(h_8\).
Find-S encuentra una hipótesis de máxima especificidad consistente con todos los ejemplos positivos que es \(h_8\):
\begin{center}
Por lo tanto, \(H = \{h_8 : (x, y) \in X \mid a, b \in \mathbb N, a \leq x < b, 8 = \frac{b(b-1)}{2} + a + 1\}\)
\end{center}
\end{itemize}

\section{Ejercicio}
En este ejercicio consideraremos \(X = \{0, 1\}^n\), i.e., \(X\) es el conjunto de todas las cadenas de longitud \(n\) formadas por ceros
y unos.
\begin{enumerate}
    \item ¿Cuantos ejemplos positivos del concepto palíndromo hay en \(X\)?
    
    Palíndromo es una palabra o frase cuyas letras están dispuestas de tal manera que resulta la misma leída de izquierda a derecha que de derecha a izquierda. Analizaré las posibles combinaciones para los distintos valores de \(n\).
    \begin{itemize}
        \item \(n = 1\)
        \begin{itemize}
            \item \(X = \{\{0\}, \{1\}\} \Rightarrow p_1 =2\)
        \end{itemize}
    \item  n = 2
        \begin{itemize}
            \item \(X = \{\{0,0\}, \{1,1\}\} \Rightarrow p_2 =2\)
        \end{itemize}
    \item n = 3
        \begin{itemize}
            \item \(X = \{\{0,0,0\},\{0,1,0\}, \{1,1,1\}, \{1,0,1\}\} \Rightarrow p_3 =4\)
        \end{itemize}
    \item n = 4
        \begin{itemize}
            \item \(X = \{\{0,0,0,0\}, \{\{1,0,0,1\}, \{\{0,1,1,0\}, \{\{1,1,1,1\}\} \Rightarrow p_4 =4\)
        \end{itemize}
    \item n = 5
        \begin{itemize}
            \item \(X = \{\{0,0,0,0,0\}, \{0,0,1,0,0\}, \{1,1,1,1,1\}, \{1,0,0,0,1\}, \{0,1,0,1,0\}, \{1,0,1,0,1\}, \{0,1,1,1,0\}, \{1,1,0,1,1\}\} \Rightarrow p_5 =8\)
        \end{itemize}
    \item n = 6
        \begin{itemize}
            \item \(X = \{\{0,0,0,0,0,0\}, \{0,0,1,1,0,0\}, \{1,1,1,1,1,1\}, \{1,0,0,0,0,1\}, \{1,1,0,0,1,1\}, \{0,1,1,1,1,0\}, \{0,1,0,0,1,0\},\{1,0,1,1,0,1\}\} \Rightarrow p_6 =8\)
        \end{itemize}
    \end{itemize}
    Por definición de palíndromo y su simetría podemos analizar únicamente la primer mitad de la cadena. Veamos cuando n es par:
    \begin{itemize}
        \item Si \(n = 2\): Las opciones son:
        \\- Cero 1's y un 0
        \\- Un 1 y cero 0's
        \\Ambas son permutaciones de elementos que se repiten. Supongamos que tenemos a veces el número 1 y b veces el número 0, tendremos una permutación de \(n = a+b\) elementos en las que el número 1 se repite \(a\) veces y el número 0 \(b\) veces. Una forma de expresarlo es:
        \\\(P_{n}^{a,b}=\frac{n!}{a!b!}, n = a+b\)
        \\Por tanto, \(p_2=P_{1}^{0,1}+P_{1}^{1,0}=\frac{1!}{0!1!}+\frac{1!}{1!0!}=2\)
        \item Si \(n = 4\):
        \\- Cero 1's y dos 0
        \\- Un 1 y un 0
        \\- Dos 1's y cero 0's
        \\Por tanto, \(p_4=P_{2}^{0,2}+P_{2}^{1,1}+P_{2}^{2,0}=\frac{2!}{0!2!}+\frac{2!}{1!1!}+\frac{2!}{2!0!}=4\)
        \item Si \(n = 6\):
        \\- Cero 1's y tres 0's
        \\- Dos 1's y un 0
        \\- Un 1 y dos 0's
        \\- Tres 1's y cero 0's
        \\Por tanto, \(p_6=P_{3}^{0,3}+P_{3}^{2,1}+P_{3}^{1,2}+P_{3}^{3,0}= \frac{3!}{0!3!}+\frac{3!}{2!1!}+\frac{3!}{1!2!}+\frac{3!}{3!0!}=8\)
        \item Si n es par por lo expuesto:
        \\- Cero 1's y \((n/2)\) 0's
        \\- Un 1 y \((n/2 -1)\) 0's
        \\- Dos 1's y \((n/2 - 2)\) 0's
        \\- \(\ldots\)
        \\- \((n/2 -2)\) 1's y dos 0's
        \\- \((n/2 - 1)\) 1's y un 0
        \\- \((n/2)\) 1's y cero 0's
        \\Por tanto, \(p_{n-par}=P_{n/2}^{0,n/2}+P_{n/2}^{2,n/2-2}+\ldots+P_{n/2}^{n/2-2,2}++P_{n/2}^{n/2-1,1}++P_{n/2}^{n/2,0}\)
    \end{itemize}
    Veamos cuando \(n\) es impar:
    \begin{itemize}
        \item Si \(n = 3\):
        \\- Cero 1's: 1 posibilidad
        \\- Un 1: 1 posibilidad
        \\- Dos 1's: 1 posibilidad
        \\- Tres 1's: 1 posibilidad
        \item Si \(n = 5\):
        \\- Cero 1's: 1
        \\- Un 1: Equivalente a \(n = 4\) con cero 1's
        \\- Dos 1's: Equivalente a \(n = 4\) con dos 1's
        \\- Tres 1's: Equivalente a \(n = 4\) con dos 1's
        \\- Cuatro 1's: Equivalente a \(n = 4\) con cuatro 1's
        \\- Cinco 1's: 1
        \item Si \(n = 7\):
        \\- Cero 10s: 1
        \\- Un 1: Equivalente a \(n = 6\) con cero 1's
        \\- Dos 1's: Equivalente a \(n = 6\) con dos 1's
        \\- Tres 1's: Equivalente a \(n = 6\) con dos 1's
        \\- Cuatro 1's: Equivalente a \(n = 6\) con cuatro 1's
        \\- Cinco 1's: Equivalente a \(n = 6\) con cuatro 1's
        \\- Seis 1's: Equivalente a \(n = 6\) con seis 1's
        \\- Siete 1's: 1
    \end{itemize}
    Por tanto, podemos verificar que al ser un numero impar, el numero del centro de la cadena no tiene ninguna restricción, Entonces para el \(n\) impar vamos a tener siempre el doble de opciones  que para \(n-1\). Osea \(p_{n-impar}=2P_{n-1}\)
    \\\\En suma, el numero de ejemplos positivos de palíndromo en \(X\) sigue la siguiente regla:
    \begin{center}
    \(P_n =
    \left\{
    	\begin{array}{ll}
    		P_{n/2}^{0,n/2}+P_{n/2}^{2,n/2-2}+\ldots+P_{n/2}^{n/2-2,2}++P_{n/2}^{n/2-1,1}++P_{n/2}^{n/2,0}  & \mbox{n } par  \\
    		2P_{n-1} & \mbox{n } impar
    	\end{array}
    \right.
    \)
    \end{center}
    
    \item Sea \(\omega\) el concepto definido en \(X\) de la siguiente manera: \(\omega(y) = 1\) si y sólo si \(y\) contiene como máximo dos \(1`\)s. Prueba que el n´umero de ejemplos positivos de \(\omega\) es una función cuadrática de \(n\).
    \\\\Por definición y por ejercicio anterior encontramos la posibilidad de:
    \begin{itemize}
        \item Cero 1's, hay 1 posible combinación.
        \item Un 1, hay \(n\) combinaciones.
        \item Dos 1's, hay \(P_{n}^{2,n-2}=\frac{n!}{2!(n-2)!}=\frac{n(n-1)}{2}\).
    \end{itemize}

     \begin{center}
         Por tanto, el numero de positivos seguirá: \(1+n+\frac{n(n-1)}{2}= \frac{1}{2}n^2+\frac{1}{2}n+1\) que efectivamente es cuadrática
     \end{center}
    
    \item Supongamos que en un problemas de aprendizaje sobre\(X\) aplicamos el algoritmo de \textit{aprendizaje por enumeración} sobre el con junto de todas las hipótesis y las hipótesis están enumeradas de manera que la que buscamos está en la primera mitad. Si podemos probar un millón de hipótesis por segundo y \(X = \{0, 1\}^9\), ¿cuánto tiempo llevará encontrar la hipótesis buscada en el peor de los casos?
    
    Utilizando la formular de permutaciones tenemos que el conjunto de \(X\) esta formado por 512 elementos \(P_{9}^{0,9}+P_{9}^{1,8}+\ldots+P_{9}^{9,1}+P_{9}^{9,0}\).
    \\Si el conjunto de hipótesis contiene todas las hipótesis, tenemos que \(H = 2^X\), lo que quiere decir que en \(H\) hay \(2^{512} \simeq 10^{153}\) hipótesis.
    \\Suponiendo que esta en la primera mitad, tenemos \(10^{153}/2 \simeq 10^{153}\).
    \\\\Si podemos probar un millón de hipótesis por segundo, necesitamos \(10^{153}/(60*10^6) \simeq 10^{145}\) segundos para probar nuestra hipótesis.
\end{enumerate}
\end{document}